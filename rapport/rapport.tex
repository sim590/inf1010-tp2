\documentclass[12pt,french]{article}

%--------------------
% Basic packages
%--------------------
\usepackage{babel}
\usepackage[utf8]{inputenc}
\usepackage[T1]{fontenc}
\usepackage{graphicx}

% LIENS
\usepackage{hyperref}
\hypersetup{hidelinks, colorlinks, citecolor=black, linkcolor=black, urlcolor=blue}

% ENTÊTES
\usepackage{fancyhdr}
\pagestyle{fancy}
\lhead{INF1010\\Travail Pratique 2}
% STYLE SECTIONS
\usepackage{sectsty}
\sectionfont{\sectionrule{0ex}{0pt}{-1ex}{0.5pt}}

\usepackage{float}

\usepackage{array}
% CODE LISTING
\usepackage{listings}

% CAPTIONS
%\usepackage{caption}
%\captionsetup[lstlisting]{ format=listing, labelfont=white, textfont=white, singlelinecheck=false, %margin=0pt, font={bf,footnotesize} }

% COULEURS
\usepackage{color}
\definecolor{mygreen}{rgb}{0,0.6,0}
\definecolor{mygray}{rgb}{0.5,0.5,0.5}
\definecolor{mymauve}{rgb}{0.58,0,0.82}

\lstset{ %
      backgroundcolor=\color{white},   % choose the background color; you must add \usepackage{color} or \usepackage{xcolor}
      basicstyle=\footnotesize,        % the size of the fonts that are used for the code
      breakatwhitespace=false,         % sets if automatic breaks should only happen at whitespace
      breaklines=true,                 % sets automatic line breaking
      captionpos=b,                    % sets the caption-position to bottom
      commentstyle=\color{mygreen},    % comment style
      deletekeywords={...},            % if you want to delete keywords from the given language
      escapeinside={\%*}{*)},          % if you want to add LaTeX within your code
      extendedchars=true,              % lets you use non-ASCII characters; for 8-bits encodings only, does not work with UTF-8
      frame=single,                    % adds a frame around the code
      keepspaces=true,                 % keeps spaces in text, useful for keeping indentation of code (possibly needs columns=flexible)
      keywordstyle=\color{blue},       % keyword style
      language=Octave,                 % the language of the code
      morekeywords={*,...},            % if you want to add more keywords to the set
      numbers=left,                    % where to put the line-numbers; possible values are (none, left, right)
      numbersep=5pt,                   % how far the line-numbers are from the code
      numberstyle=\tiny\color{mygray}, % the style that is used for the line-numbers
      rulecolor=\color{black},         % if not set, the frame-color may be changed on line-breaks within not-black text (e.g. comments (green here))
      showspaces=false,                % show spaces everywhere adding particular underscores; it overrides 'showstringspaces'
      showstringspaces=false,          % underline spaces within strings only
      showtabs=false,                  % show tabs within strings adding particular underscores
      stepnumber=2,                    % the step between two line-numbers. If it's 1, each line will be numbered
      stringstyle=\color{mymauve},     % string literal style
      tabsize=2,                       % sets default tabsize to 2 spaces
      title=\title                   % show the filename of files included with \lstinputlisting; also try caption instead of title
}
% Règle les problèmes d'encodage de caractères dans les fichiers sources
\lstset{literate=%
{á}{{\'a}}1 {é}{{\'e}}1 {í}{{\'i}}1 {ó}{{\'o}}1 {ú}{{\'u}}1
{Á}{{\'A}}1 {É}{{\'E}}1 {Í}{{\'I}}1 {Ó}{{\'O}}1 {Ú}{{\'U}}1
{à}{{\`a}}1 {è}{{\'e}}1 {ì}{{\`i}}1 {ò}{{\`o}}1 {ò}{{\`u}}1
{À}{{\`A}}1 {È}{{\'E}}1 {Ì}{{\`I}}1 {Ò}{{\`O}}1 {Ò}{{\`U}}1
{ä}{{\"a}}1 {ë}{{\"e}}1 {ï}{{\"i}}1 {ö}{{\"o}}1 {ü}{{\"u}}1
{Ä}{{\"A}}1 {Ë}{{\"E}}1 {Ï}{{\"I}}1 {Ö}{{\"O}}1 {Ü}{{\"U}}1
{â}{{\^a}}1 {ê}{{\^e}}1 {î}{{\^i}}1 {ô}{{\^o}}1 {û}{{\^u}}1
{Â}{{\^A}}1 {Ê}{{\^E}}1 {Î}{{\^I}}1 {Ô}{{\^O}}1 {Û}{{\^U}}1
{œ}{{\oe}}1 {Œ}{{\OE}}1 {æ}{{\ae}}1 {Æ}{{\AE}}1 {ß}{{\ss}}1
{ç}{{\c c}}1 {Ç}{{\c C}}1 {ø}{{\o}}1 {å}{{\r a}}1 {Å}{{\r A}}1
{€}{{\EUR}}1 {£}{{\pounds}}1
}


\begin{document}
    \begin{titlepage}
        \begin{center}
            \noindent\rule{13cm}{1pt}\\[0.4cm]
             % Titre
            \textsc{\huge \bfseries Travail Pratique 2}\\
                                    INF1010\\[0.4cm]
            \noindent\rule{13cm}{1pt}\\[5cm]
            
            % Auteurs
            \begin{minipage}{0.4\textwidth}
                \begin{flushleft}
                \large\emph{Auteur(s):}\\[0.5cm]
                    Simon \textsc{Désaulniers}\\
                    Frédéric \textsc{Hamelin}
                \end{flushleft}
            \end{minipage}
            \begin{minipage}{0.5\textwidth}
                \begin{flushright} \large
                    \emph{Professeur:} \\[0.5cm]
                    Boucif \textsc{Amar Bensaber}, Ph.D
                    \vspace{\parskip}
                \end{flushright}
            \end{minipage}
    
            % On va au bas de la page
            \vfill
            {\large Université du Québec à Trois-Rivières\\ \today}
        \end{center}
        \thispagestyle{empty}
    \end{titlepage}
    \pagenumbering{roman}
    \setcounter{page}{1}

    \tableofcontents
    \newpage
    
    \pagenumbering{arabic}
    \setcounter{page}{1}

    \section{Introduction} % (fold)
    \label{sec:intro}
        La pile {\tt TCP/IP} permet une abstraction de la gestion de communication sur un réseau
        d'ordinateur. Ce travail permettera de démontrer la compréhension chez l'étudiant des outils
        de développement d'applications communicant sur le réseau.

        \subsection{Objectifs et buts} % (fold)
        \label{sub:obj-buts}
            Dans le cadre du cours INF1010, le travail pratique 2 vise à créer une application du type
            client/serveur permettant la communication entre plusieurs clients en mode {\tt TCP}. La
            gestion de {\tt sockets} et de connexions multiples doit être faite de façon à ce que
            plusieurs clients communiquent à l'aide du service offert par le serveur. Plus précisément,
            plusieurs utilisateurs doivent avoir le pouvoir de :
            \begin{itemize}
                \item communiquer au moyen de texte (<< chat >>);
                \item envoie de requêtes spéciales :
                    \begin{itemize}
                        \item connexion et déconnexion;
                        \item liste des clients connectés;
                        \item envoie de message à un client, un sous-ensemble de la totalité et la
                            totalité;
                    \end{itemize}
            \end{itemize}
        % subsection obj_buts (end)
    % section Introduction (end)

    \section{Méthodologie de conception} % (fold)
    \label{sec:method-concept}

        % conception client-serveur
        \subsection{Échange client/serveur} % (fold)
        \label{sub:echange client/serveur}
            Afin de communiquer de façon uniforme le client et le serveur ont étés prévu pour
            échanger des structures encapsulant l'information. Ces structures comprennent entre
            autres 
        % subsection echange client/serveur (end)

        \subsection{Client} % (fold)
        \label{sub:client}
%- msg::
    %description: Envoie un message à l'utilisateur spécifié.
    %synopsis: /msg <nom_utilisateur> <message>
     %- nom_utilisateur: Le nom d'utilisateur de la personne à qui envoyer le
                        %message. Si le nom d'utilisateur est "-", envoyer à tous
                        %les clients connectés.
     %- message: Le message à envoyer.

%- names::
    %synopsis: /names
    %description: Requête de la liste des clients connectés dans le canal.

    %Affichage à l'écran (suggestion):

    %Clients dans votre canal:
    %-------------------------
    %[simon] [fred] [patate_poel] ...
    %[mathieu] [simon2] [balbal] ...
    %...

%- list::
    %description: Requête de la liste de canaux disponibles.
    %synopsis: /list

    %Affichage à l'écran (suggestion):

    %Canaux ouverts sur le serveur:
    %------------------------------
    %[<nom_canal>]: <topic>
    %[<nom_canal>]: <topic>
    %[<nom_canal>]: <topic>
    %...

%- topic::
    %description: Requête du topic du canal.
    %synopsis: /topic

%- connect::
    %description: Connecte le client au serveur spécifié.
    %synopsis: /connect <nome_hôte>
     %- nome_hôte: Le nom d'ĥôte du serveur auquel se connecter.

%- join::
    %description: Change le canal dans lequel le client clavarde.
    %synopsis: /join <nom_canal>
     %- nom_canal: Le nom du canal auquel se joindre.

%- disconnect::
    %description: Déconnecte le client du serveur.
    %synopsis: /disconnect

%- quit::
    %description: Ferme le programme.
    %synopsis: /quit

    %C'est la combinaison d'un /disconnect avec la fermeture du programme. Par
    %conséquent, le serveur s'attendra à reçevoir un paquet "/disconnect".
        
        % subsection client (end)

        \subsection{Serveur} % (fold)
        \label{sub:serveur}
            
        % subsection serveur (end)

        \subsection{Croquis, organigramme ou algorithmes} % (fold)
        \label{sub:croquis-org-algo}
        
        % subsection croquis-org-algo (end)

        \subsection{Structures de données} % (fold)
        \label{sub:struct-donnes}
        
        % subsection struct-donnes (end)
    % section method-concept (end)

    \section{Analyse et description du programme} % (fold)
    \label{sec:analyse-desc}
    
    % section analyse-desc (end)

    \section{Discussions} % (fold)
    \label{sec:discussions}
    
    % section discussions (end)

    \section{Conclusion} % (fold)
    \label{sec:Conclusion}
    
    % section Conclusions (end)

    % ----------
    % En annexe:
    % ----------
    %
    % Code source
    % manuels
\end{document}
